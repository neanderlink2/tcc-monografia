\chapter{Introdução}
A área de Desenvolvimento de Sistemas é uma área relativamente nova, mas que atualmente impacta em toda e qualquer área do conhecimento. Desde pequenos sistemas de gerenciamento de tarefas a grandes redes sociais, os Sistemas de Informação estão em cada momento de nossas vidas e inclusive, na educação.

\section{Tema}
As ferramentas disponíveis na Web estão cada vez mais sofisticadas e completas, e dispensando o uso de papéis ou de programas instalados fisicamente em suas máquinas, além de possuírem armazenamento compartilhado em nuvem. A utilização dessas ferramentas trazem mais praticidade e conforto ao usuário, principalmente para estudantes por manter todos os seus trabalhos, apresentações, anotações salvas na nuvem e acessíveis por qualquer dispositivo.

\section{Problema}
Atualmente, quando inicia-se os estudos com uma linguagem de programação, os instrutores geralmente utilizam uma pseudo-linguagem para ensinar os básicos de algoritmos.
Em geral, os ambientes de desenvolvimento dessas pseudo-linguagens são simples e fáceis de usar, com apenas alguns cliques.

Entretanto, durante a transição para uma linguagem de programação real (como Java, C\#, Python, etc...) podem ocorrer problemas na execução do código. Em alguns casos é necessário instalar um Kit de Desenvolvimento, ou um ambiente de desenvolvimento que torna complexo a sua instalação. Por conta disso, alguns alunos se desmotivam a aprender a linguagem ao passarem mais tempo tentando executar o código do que desenvolvendo alguma atividade.

\subsection{Objetivo geral}
Visando facilitar o início da aprendizagem com uma nova linguagem de programação uma ferramenta Web pode ajudar nessa transição para evitar que o estudante desista ainda na ambientação, além de motivá-lo a continuar estudando algoritmos sem precisar ter uma máquina potente ou um ambiente instalado em qualquer lugar.

\subsection{Objetivos específicos}
O sistema deverá:
\begin{itemize}
	\item Permitir ao usuário escolher a linguagem de programação desejada para resolver os algoritmos.
	\item Comparar suas estatísticas com a de outros usuários, com o intuito de criar um pequeno ambiente de competição.
	\item Possibilitar o usuário a desenvolver seus algoritmos sem preocupar-se com a ambientação da linguagem.
	\item Disponibilizar uma forma de criar um Avatar para sentir-se mais próximo dentro da plataforma.
\end{itemize}


\section{Estrutura do TCC}
Este trabalho consistirá em uma documentação para a Ferramenta de Desenvolvimento de Software na Web.
Inicialmente haverá uma breve revisão de sistemas similares disponíveis no mercado e logo após se iniciará a documentação do sistema, com o Documento de Visão, Descrição resumida dos casos de uso, diagramas de caso de uso, descrição detalhada dos casos de uso, protótipos de interfaces do sistema, diagramas de classe, diagramas de sequência e casos de teste.
Alguns desses documentos estarão em outros arquivos fora deste, cada seção definirá como será descrita.

\subsection{Classificação da Pesquisa}
Esta pesquisa será descritiva onde procurará identificar os fatores que envolvem os estudantes e a forma como eles aprendem. Também tem como objetivo documentar e explicar como o sistema será desenvolvido.
Será desenvolvido um sistema Web, utilizando a biblioteca React JS para as interfaces de usuário e a plataforma ASP.NET Core 2.2 para desenvolvimento dos serviços disponibilizados.
