\chapter{Introdução}
A área de Desenvolvimento de Sistemas é uma área relativamente nova, mas que atualmente impacta em toda e qualquer área do conhecimento. Desde pequenos sistemas de gerenciamento de tarefas a grandes redes sociais, os Sistemas de Informação estão em cada momento de nossas vidas e inclusive, na educação.

\section{Tema}
As ferramentas disponíveis na Web estão cada vez mais sofisticadas e completas, e dispensando o uso de papéis ou de programas instalados fisicamente em suas máquinas, além de possuírem armazenamento compartilhado em nuvem. A utilização dessas ferramentas trazem mais praticidade e conforto ao usuário, principalmente para estudantes por manter todos os seus trabalhos, apresentações, anotações salvas na nuvem e acessíveis por qualquer dispositivo.


\section{Problema}
Atualmente, quando se começa a aprender uma linguagem de programação, os instrutores geralmente começam com uma pseudo-linguagem para ensinar os básicos de algoritmos. Em geral, os ambientes dessas pseudo-linguagens são simples e fáceis de usar, com apenas alguns cliques.

Entretanto, quando passa-se para uma linguagem de programação mesmo (como Java, C\#, Python, etc...) sempre há problemas na primeira vez que executamos. Variáveis de ambiente, máquina mal ambientada ou problemas de hardwares são alguns dos fatores que podem atrapalhar na hora de estudar uma linguagem de programação robusta.

\subsection{Objetivo geral}
Visando facilitar o início da aprendizagem com uma nova linguagem de programação uma ferramenta Web pode ajudar nessa transição para evitar que o estudante desista ainda na ambientação, além de motivá-lo a continuar estudando algoritmos sem precisar ter uma máquina potente ou um ambiente instalado em qualquer lugar.

\subsection{Objetivos específicos}
Os objetivos específicos detalham os objetivos gerais através de etapas ou fases de pesquisa. Devem ser utilizados verbos no infinitivo, assinalando as ações propostas para alcançar o objetivo geral. Os verbos utilizados aqui são os de ação, que serão utilizados na metodologia.

\section{Estrutura do TCC}
Este trabalho consistirá em uma documentação para a Ferramenta de Desenvolvimento de Software na Web.
Inicialmente haverá uma breve revisão de sistemas similares disponíveis no mercado e logo após se iniciará a documentação do sistema, com o Documento de Visão, Descrição resumida dos casos de uso, diagramas de caso de uso, descrição detalhada dos casos de uso, protótipos de interfaces do sistema, diagramas de classe, diagramas de sequência e casos de teste.
Alguns desses documentos estarão em outros arquivos fora deste, cada seção definirá como será descrita.

\subsection{Classificação da Pesquisa}
Este trabalho consiste em um Trabalho de Conclusão de Curso focado em implementação de um Sistema sob demanda.
