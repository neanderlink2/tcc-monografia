\chapter{Trabalhos correlacionados}
\label{chp:conceitos-gerais}
Conforme explorado no capítulo \ref{chp:introducao}, um ambiente de desenvolvimento na Web é essencial para o engajamento de estudantes na área de desenvolvimento de sistemas. Com isso, nossa ferramenta Web de desenvolvimento pretende disponibilizar algoritmos e um ambiente de desenvolvimento on-line, com a possibilidade de salvar os algoritmos na nuvem e permitirá validar eles. Os usuários também poderão analisar onde estão acertando e errando para que possam se auto-avaliar.

A disponibilização de ambientes de desenvolvimento na Web não é novidade. Entretanto, cada ferramenta tem sua especialidade e foco em uma área de desenvolvimento. Neste capítulo serão apresentados sistemas que seguem essa ideia. Após, haverá uma tabela apontando as diferenças entre cada plataforma.

\section{CodePen}
\label{sec:codepen}
O CodePen é um ambiente de desenvolvimento colaborativo com foco no desenvolvimento de Front-end e de código aberto. O CodePen foi desenvolvido como uma comunidade de compartilhamento de demonstrações e exemplos. Ele permite que os usuários desenvolvam sites através do navegador. 

Ele é ideal para expor trabalhos interativos e funcionais \cite{fiala2016collaborative}. As principais vantagens do CodePen se dão em seu ambiente de desenvolvimento dinâmico em que as alterações aparecem na página logo após editar um arquivo HTML.

Como os dados estão salvos em nuvem, a disponibilização de código fica mais rápida e simples, permitindo que desenvolvedores possam expor portfólios editáveis, ao mesmo tempo que o deixam disponíveis para que outras pessoas possam utilizar essa página como inspiração.

\section{JSFiddle}
\label{sec:jsfiddle}
O JSFiddle também é um ambiente de desenvolvimento colaborativo. Tal qual o CodePen \ref{sec:codepen}, permite manter códigos armazenados e compartilhá-los com outros desenvolvedores. Entre os diferenciais do JSFiddle, está o IntelliSense, que apresenta uma lista de possibilidades de códigos para serem escritos.

O JSFiddle é diferente pois ele também permite a disponibilização de códigos, entretanto não tem um foco em criar uma comunidade de desenvolvedores. Uma das diferenças do JSFiddle é a possibilidade de utilizar $boilerplates$, códigos já prontos com o mínimo de configuração inicial para que seja executado com sucesso.

Essa ferramenta possui a tecnologia do IntelliSense, uma espécie de "autocomplete" que surgere possíveis códigos para que o desenvolvedor complete seu código rapidamente e até mesmo lembre de como uma função ou objeto funciona.

\section{Coding Ground}
\label{sec:codingground}
O Coding Ground é uma plataforma de desenvolvimento para sites de tutoriais. Ele disponibiliza um pequeno ambiente para que o usuário possa editar códigos e executá-los. Também possui uma vasta biblioteca de linguagens possíveis para utilização, inclusive com frameworks e bibliotecas de Front-End.

O Coding Ground se distancia um pouco das plataformas que serão discutidas nessa seção, pois é uma plataforma para criação de tutoriais, mas que permite a utilização de um ambiente de desenvolvimento que funciona com precisão e velocidade.

Essa ferramenta não é capaz de executar testes unitários e nem possui nenhuma forma de desafios ou algoritmos para serem resolvidos.

\section{Fabriki}
\label{sec:fabriki}
Uma plataforma educacional com foco em ensinar linguagem de programação. Criada para guiar o estudante do iniciante ao avançado. Possui suporte para desenvolvimento Web e algoritmos em Java. A plataforma também possibilita a validação dos algoritmos quando submetidos.

O grande diferencial do Fabrik a outras plataformas é que os algoritmos não ficam presos a apenas métodos ou classes, ele é capaz de executar testes unitários nos códigos enviados que permitem exercícios complexos como algoritmos de Orientação a Objetos complexos e desenvolvimento de Back-End utilizando a linguagem Java.

Além disso, os algoritmos disponíveis ficam separados por turmas e disponibilizados aos alunos aos poucos. Os algoritmos disponíveis são explicativos e podem possuir até mesmo vídeos e dicas de como resolver os algoritmos.

\section{The Huxley}
\label{sec:thehuxley}
Uma plataforma competitiva, que possibilita o desenvolvimento direto pelo Browser, com validação de códigos e algoritmos desafiadores. Esta plataforma possui suporte para  até 7 linguagens de programação que o usuário pode escolher ao começar. Está disponível gratuitamente.

É uma ótima ferramenta para sala de aula, pois permite criar salas e disponibilizar algoritmos para serem resolvidos, rapidamente e com testes locais e submissão a testes. 

Também possui uma tela com estatísticas do algoritmo, onde o pessoal mais acerta, erra, se há erro de compilação, entre outros problemas.

Apesar de bem completo, seu ambiente é um pouco confuso e que pode afastar usuários que não se apeguem à plataforma.




\begin{table}[htb]
	\caption{Tabela comparativa entre os trabalhos correlacionados}
	\footnotesize
	\label{tb:trabalhos-correlacionados}
	\centering
	\begin{tabular}{|l|c|c|c|c|} %left, center, right. Você pode mudar isso
		\hline
        Sistema/Ferramenta & Salva na nuvem & Validação de Código & Colaboração ao Vivo & Compartilhamento \\
		\hline
		CodePen & Sim & Não & Não & Sim \\
		\hline
		JSFiddle & Sim & Não & Sim & Sim \\
		\hline
		Coding Ground & Sim & Não & Não & Sim \\
		\hline
		Fabriki & Não & Sim & Não & Não \\
		\hline
		The Huxley & Não & Sim & Não & Não \\
		\hline
	\end{tabular}
\end{table}
